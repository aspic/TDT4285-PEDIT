\section{Testing}
\subsection{Sending mail}
This section will test whether the mail sending actually works.
\subsubsection{Input}
Start mutt (a text based mail sending client) on the server, write a
mail and send to both an internal and external address.
\begin{lstlisting}
mutt # starts mutt
press 'm' # creates a new mail
To: jonasei@tdt4285.idi.ntnu.no, kjetil861@gmail.com # add addresses
Subject: test # add subject
press ':', write 'wq' # save mail
press 'y' # send the mail
\end{lstlisting}
\subsubsection{Expected output}
A mail will be received at the two valid addresses
jonasei@tdt4285.idi.ntnu.no (internal) and kjetil861@gmail.com
(external).
\subsection{Receiving mail}
Send a mail from an external address. By default Exim delivers internal
mail so that function is not necessary to test.
\subsubsection{Input}
Create and send a mail from an external mail client like Gmail, Yahoo or
another mail provider. Send mail to kjetime@tdt4285.idi.ntnu.no.
\subsubsection{Expected output}
A new mail will be delivered to the user, kjetime.
\subsection{Spam}
This section details testing the zen-spamhaus functionality of Exim.
\subsubsection{Input}
To test the spamhaus functionality you need to use the exim4 CLI as detailed below.
\begin{lstlisting}
# exim4 -bhc 192.203.178.178
helo pbl.crynwr.com
mail from:asdf@pbl.crynwr.com
rcpt to:kjetime@tdt4285.idi.ntnu.no
\end{lstlisting}

\subsubsection{Expected output}
\begin{lstlisting}
>>> deny: condition test succeeded
550-X-Warning: 192.293.178.178 is listed at zen-spamhaus.itea.ntnu.no
...contd
\end{lstlisting}
