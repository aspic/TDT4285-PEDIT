\section{Installation process}
This section describes the installation process, what was configured and
what choices we made regarding the setup. This section should serve as a
``stand alone'' guide, that will be sufficient when setting up a
identical system from scratch.
\subsection{Pre-conditions}
The following guide assumes that these conditions has been met:
\begin{itemize}
	\item The server is running Debian Squeeze, installed with the base
system
	\item The acting user has root access
	\item A text editor is installed (we assume Vim)
\end{itemize}
\subsection{Install procedure}
This section will describe the installation procedure step by step.
\subsubsection{Exim}
Exim4 is bundled with the base system of Debian Squeeze, but it does not
natively support authentication via libpam. For this we need to install
exim4-deamon-heavy which has built-in support. After this we only have to
reconfigure exim4.
\begin{lstlisting}
sudo aptitude install exim4-deamon-heavy
sudo dpkg-reconfigure exim4-config #Start reconfigure interface
\end{lstlisting}
Through this interface the Exim server will be reconfigured to suit our
needs. Each dialogue (in order will be described below), and what option
we chose will be highlighted. In the first dialogue, press
\emph{OK}.
\begin{enumerate}
\item\textbf{General type of mail configuration:}\\
By default only local delivery works in Exim. Since we want to
send/receive external mail, and since the mail is sent my the NTNU
gateway, choose: \emph{mail sent by smarthost; received via SMTP or
fetchmail}.
\item\textbf{System mail name:}\\
Enter \emph{tdt4285.idi.ntnu.no}.
\item\textbf{Exim IP configuration warning}\\
Read through the information and press \emph{OK}.
\item\textbf{Enter IP-address to listen on for incoming SMTP
connections}\\
The IP address can be found in several ways. We execute nslookup:
\begin{lstlisting}
nslookup tdt4285.idi.ntnu.no

Server:		129.241.0.200
Address:	129.241.0.200#53

Non-authoritative answer:
Name:	tdt4285.idi.ntnu.no
Address: 129.241.106.91
\end{lstlisting}
Enter the address the \emph{129.241.106.91}.
\item\textbf{Other destinations for which mail is accepted}\\
Enter \emph{tdt4285.idi.ntnu.no}.
\item\textbf{Machines to relay mail for:}\\
Leave this field empty.
\item\textbf{IP address or host name for the outgoing smarthost:}\\
We need to define the outgoing smarthost for SMTP. Since NTNU has closed
port 25 for outgoing traffic. We will us the main mail gateway which
NTNU provided. Enter \emph{mailgw.ntnu.no}.
\item\textbf{Dial-on-demand}\\
Since our server has ``always-on'' internet access, answer \emph{NO}.
\item\textbf{Delivery method for local mail:}\\
Mbox uses a single file for the complete mail folder, while the
Maildir-approach split mails and place them in a folder located in
$\sim$/Maildir. Choose the second method \emph{Maildir format}.
\item\textbf{Configuration format:}
In this step choose to split the configuration file into several other
smaller files
\end{enumerate}

\subsubsection{Handling spam}
To handle SPAM we will use the DNSBL-servers from NTNU IT which is free of use
for any server situated on campus. This needs to be handled by Exim.

\begin{enumerate}
  \item Edit \emph{/etc/exim4/conf.d/main/02\_exim4-config\_options} and add the following at the end of the file:
\begin{lstlisting}
# Add Spamhaus protection
CHECK_RCPT_IP_DNSBLS = zen-spamhaus.itea.ntnu.no}
\end{lstlisting}

\item Edit \emph{/etc/exim4/conf.d/acl/30\_exim4-config\_check\_rcpt} and search for DNSBLS and substitute 'warn' for 'deny'.

\item Restart Exim using \emph{/etc/init.d/exim4 restart}
\end{enumerate}

Exim is now fully configured.

\subsubsection{Dovecot}
Dovecot will be our IMAP service and is located in the package
repository in Debian Squeeze. First, install the dovecot package.
\begin{lstlisting}
sudo aptitude install dovecot-imapd
\end{lstlisting}
Most parts of the dovecot default configuration will suffice, we only
need to tell it where to find the user mail.
\begin{enumerate}
	\item\textbf{Open the configuration file:}\\
 	\emph{sudo vim /etc/dovecot/dovecot.conf}
	\item\textbf{Locate the maildir section, and set correct maildir:}\\
	Put \emph{mail\_location = maildir:$\sim$/Maildir} in the
	configuration file.
\end{enumerate}
Dovecot is now fully installed and configured.

