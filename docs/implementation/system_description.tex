\section{System description}
This section describes the different components used to fulfill and
execute the planning phase. We will briefly mention the software and
protocols used, and how they are integrated into the server.
\subsection{Software}
The following software was used to create our Email service.
\subsubsection{Exim}
Exim, our (E)SMTP server, was installed through apt-get and it had a simple
'plug-and-play' installation guide. There was some automatization and
configuration issues regarding standardizing the defualt port: We were
not able to use port 25, which is the default outgoing port for any mail
service because NTNU has disabled this port. \\
We choose to use NTNU's mailgateway \emph{mailgw.ntnu.no} for our
outgoing port, rather then using a custom port through
\emph{tdt4285.idi.ntnu.no}\\ A consequence of this will be that the user
will have to specify the outgoing port in their Email client manually,
then rather using the defualt adress \emph{tdt4285.idi.ntnu.no:25}.
Since port 25 would have never worked in the first place, it was
inevitable that some client-side configuration was necessary.
\subsubsection{Dovecot}
Dovecot, the IMAP server was in the package repository, hence we had
only to install it. This was not an issues since Dovecot is a simple
'plug-and-play' application that requires minimal configuration. \\
We were able to use the default port 143 for incoming mail, in any Email
client by connecting to the server with:
\emph{username@tdt4285.idi.ntnu.no:143}.
\subsection{Protocols}
\subsubsection{IMAP}
Internet message access protocol (IMAP) is the standard protocol for
mail retrieval which is implemented in the application layer. It has a
simple set of messages and commands that can be sent and recieved
between the client-application and the email server. These messages and
commands allows a client to remotely access email on an email server by
provding a valid username and password.
\subsubsection{SMTP}
The Simple Mail Transfer Protocol (SMTP) handles mail submissions by the
client and makes them accessible to the recieving client to retrive it
via the IMAP protocol. It uses several intermediate server to complete
this processes, including a mail submission agent (MSA) server and a
mail transfer agent (MTA).
\subsection{Other}
\subsubsection{Monitoring}
During the implementation we used simple monitoring tools like \emph{tail -f} 
to read any error messages during an email submission or retrival (mail.err). 
As well as to confirm that that our mail server accepted the email (mail.log).\\


