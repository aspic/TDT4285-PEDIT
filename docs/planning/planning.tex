% This template from http://www.vel.co.nz

\documentclass[12pt]{article}
\usepackage{geometry}
\usepackage{graphicx}
\usepackage{float}
\usepackage{wrapfig}
\usepackage{url}
\graphicspath{{./Pictures/}}
\geometry{a4paper}
\newcommand{\HRule}{\rule{\linewidth}{0.5mm}} % command to make the lines in title page
\linespread{1.2}

%%% BEGIN DOCUMENT
\begin{document}
% Define some custom commands
\newcommand{\essayno}[0] {
	\textsc{\Large Obligatory essay no. 1}\\[0.5cm]
}
\newcommand{\essaytitle}[0] {
	\HRule \\[0.4cm]
	{ \huge \bfseries Documentation}\\[0.4cm]
	 
	\HRule \\[1.5cm]
}
 % 
% This template from http://www.vel.co.nz

\begin{titlepage}

\begin{center}
 
% Upper part of the page

\textsc{\LARGE Norwegian University of Science and Technology}\\[1.5cm]

 
\textsc{\Large Planning phase}\\[0.5cm]

\textsc{\large TDT4285 Planlegging og drift av IT-systemer}\\[0.5cm]
 
 
% Title
\HRule \\[0.4cm]
{ \huge \bfseries Service: Email}\\[0.4cm]
 
\HRule \\[1.5cm]
 
% Author

\begin{center} \Large
\emph{Authors:}\\
Kjetil \textsc{Mehl}\\
Dag-Inge \textsc{Aas}\\
Jonas \textsc{Eikli}\\[3cm]
\end{center}
 
% Bottom of the page

{\large \today}\\[4cm] % \today inputs the current date, can be changed to any other date
 
\vfill
\end{center}

\end{titlepage}
 % TITLE PAGE

\tableofcontents

\section{Planning Phase}

We are to setup and configure an email server.

The goal of this document is to reflect the choices of technology, how
we plan to meet the requirements and what we need in order to meet them.
The document is split into three parts. Discussion and planning around
the SLA, our proposed solution and what that solution may depend on.

\subsection{SLA (Service level agreement)}
The service level agreement for this service is split into two parts,
namely the general SLA (which all groups follow), and a service specific
SLA for the email-part. Both parts are discussed and evaluated below.

\subsubsection{The general SLA}
All requirements in the general SLA seem to be feasible. Many of the
requirements (like availability, performance and reliability) is
directly dependent on the operating system which our service will be
running on. 


\subsubsection{The service specific SLA}
Our service shall and must support:
\begin{itemize}
\item{Email delivery from the whole world with ESMTP.}
\item{Email sending from the system.}
\item{Email folders readable by IMAP.}
\item{Spam filtering/limiting incoming mail.}
\end{itemize}

These requirements are typical requirements for an arbitrary email
server. There exist lots of software to fulfill these requirements. The
hardest part is to decide on what software we want to use. We try to
keep the number of services required as low as possible, so that the
implementation and maintenance will be a bare minimum.

\begin{description}
\item[Performance] \hfill \\
All requirements regarding performance seems to be feasible. 10 seconds
as a time threshold for outgoing and incoming mail is highly achievable.
We will use spam-detection provided by ITEA. We believe that this
solution will detect at least 70\% of all incoming spam, and then the
service will deliver that spam to a special folder for the user.

\item[Security] \hfill \\
The mail server will be forced to only allow outgoing mail from users on
the system. The authentication will be through LDAP, which is the basic
authentication mechanism on the server.

\end{description}
\subsection{Solution}

We have chosen to implement Exim\cite{exim} as the Mail Transfer Agent. Exim has
support for ESMTP, and is licensed with the GNU General Public
License\cite{gnu-license}. We chose this MTA because the group has some previous
experience in implementing and configuring Exim. Exim also supports
everything needed by the SLA, and supports many different authentication
methods, such as LDAP and SQL. 

For IMAP, we have chosen dovecot. Dovecot is an open source IMAP and
POP3 email server for Linux/UNIX-like systems, written with security
primarily in mind\cite{dovecot}. We chose this because it offers both the mbox and
Maildir directive, in addition to POP3 and IMAP mail delivery. In
addition, the group has previous experience implementing and configuring
dovecot for mail delivery and it plugs directly into the SMTP backend
of Exim.

For spam protection, we will use the RBL-services (or DNSBL) of ITEA,
which are free since our server is located on campus. This can be
configured by following \url{http://www.exim.org/howto/rbl.html} and the URL
of this service is \url{zen-spamhaus.itea.ntnu.no}.

If this service for some reason will be unavailable or unsuitable for
this project, a fall back service will be
SpamAssassin\cite{spamassassin}. SpamAssassin is highly configurable,
can be run for the mail server, or individual users. The system is also
widely used in the industry, thus a lot of setup guides and
troubleshooting guides exist.

\subsection{Dependencies}
Both Exim and Dovecot depend on a Unix environment to run. For this, we
depend on the OS-group to implement a Unix environment, preferably
debian-based.
We also need the server to be located at NTNU, because of the
spam-protection service provided by ITEA. 

\newpage

\begin{thebibliography}{9}




\end{thebibliography}
\end{document}
